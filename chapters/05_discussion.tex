\chapter{Discussion and Outlook}%
\label{cha:discussion}

Outliers and outlier detection have been an important topic in statistics for
centuries~\citep{zimekThereBackAgain2018}. Especially in recent years work has
been done to develop and use generative models to improve the performance of
classifiers on outlier
data~\citep{meinkeNeuralNetworksThat2019,schottFirstAdversariallyRobust2018,leeTrainingConfidencecalibratedClassifiers2018}.
In this work we introduced normalizing flows for this task, with and without
the addition of archetypal analysis.

We first investigated sampling from an outlier distribution in the latent space
of a normalizing flow on toy data. In \autoref{sec:toy_example} we see, that
outlier data sampled around the normal distribution in latent space, e.g. by
using a Gumbel distribution, encircles the data well when mapped back into data
space. There, we also showed that using these samples we can train a classifier
to give low confidence scores to outlier data while keeping up the
classification performance.

Moving on to image data we showed how samples drawn from a spherical shell of
larger radius around the inlier data in latent space led to sensible images and
outliers similar to inlier images. We were able to show this on both, the
digits dataset as well as the people dataset in
\autoref{sec:qualitative_comparison}. On the people dataset we also
observed images from different classes being generated as outliers of other
classes. This again confirmed that the latent space is arranged well for
sampling from outlier distributions.

Indeed, investigating the latent space more closely in
\autoref{sec:analysis_of_the_latent_space} showed this too. Classes that are
expected to be close to each other or even hardly distinguishable, such as
\texttt{0} and \texttt{o}, are close to each other in latent space.

Archetypal analysis used in conjunction with a normalizing flow has been shown
to be not only an interesting way of structuring the latent space but also to
exert more control over the kind of outlier that is generated. After showing
that deep archetypal analysis can find meaningful extremes, such as the
different emotional expressions in the people dataset, see
\autoref{fig:ferg_aa_mse} and \autoref{fig:ferg_aa_closest}, we showed how
generating new samples from archetypes creates inlier samples as well as new
outliers with control over what visual features are extreme for the digits
dataset as well as the people dataset, see \autoref{fig:aa_emnist} to
\autoref{fig:aa_ferg_corners}. We further showed how sampling from the
nullspace of the archetype mapping creates better reconstructions and more
variation in generated images in \autoref{fig:aa_nullspace} and
\autoref{fig:aa_nullspace_bias} while allowing a lower number of archetypes,
which allows the archetypes to be more interpretable.

Finally in \autoref{sec:discriminator_performance} we were able to show that
using our methods of outlier generation can be used to train a cheaper
discriminator model and also to create classifiers, that give less false
high-confidence scores to outlier samples.

% TODO: Conclusion

\section{Future Work}%
\label{sec:future_work}


